%! Compiler = xelatex --shell-escape
%! BibTeX Compiler = biber
% !TEX TS-program = xelatex
% !TEX encoding = UTF-8
%! Author = Alan Szmyt
%! Date = 4/21/23

%! region Preamble.
\documentclass[12pt, a4paper]{extarticle}

%! region Latex Compiler Specific Configuration.
\usepackage{iftex}
\ifPDFTeX
\usepackage[utf8]{inputenc}
\usepackage[T1]{fontenc}
\usepackage{lmodern}
\else
\ifXeTeX
\usepackage{fontspec}
\setmainfont{Bitter}[
    Path=./resources/fonts/Bitter/,
    Scale=0.85,
    Extension = .ttf,
    UprightFont=*-Regular,
    BoldFont=*-Bold,
    ItalicFont=*-Italic,
    BoldItalicFont=*-BoldItalic
]
\else
\usepackage{luatextra}
\fi
\defaultfontfeatures{Ligatures=TeX}
\fi
%! endregion

%! region Packages
\usepackage[a4paper, margin=2.9cm, headsep=10pt, headheight={61.24997pt}]{geometry}
\usepackage{babel}
\usepackage{float}
\usepackage{lipsum}
\usepackage{jupynotex}
\usepackage[outputdir=../.cache/latex/out]{minted}
\usepackage{amssymb}
\usepackage{latexsym}
\usepackage{amsmath}
\usepackage{booktabs}
\usepackage{longtable}
\usepackage[obeyspaces,spaces]{url}
\usepackage{lastpage}
\usepackage{fancyhdr}
\usepackage{setspace}
\usepackage[skip=5pt plus1pt, indent=0pt]{parskip}
\usepackage{titlesec}
\usepackage{tabularray}
\usepackage{caption}
\usepackage{listings}
\usepackage[shortlabels]{enumitem}
\usepackage{xspace}
\usepackage{polyglossia}
\usepackage[hidelinks]{hyperref}
\usepackage{bookmark}
\usepackage{xcolor}
\usepackage{soul}
\usepackage{graphicx}
\usepackage{wasysym}
\usepackage{subfiles}
\usepackage{titling}
\usepackage{xargs}
\usepackage{etoolbox}
%! endregion

%! region Document Configuration.
% Set language.
\setmainlanguage{english}

% Setting graphics path to image/figures.
\graphicspath {{resources/graphics/}}

% Adjust the title up.
\setlength{\droptitle}{-4em}

\apptocmd{\thebibliography}{\raggedright}{}{}
%! endregion

%! region Colors.
\definecolor{dkgreen}{rgb}{0,0.6,0}
\definecolor{gray}{rgb}{0.5,0.5,0.5}
\definecolor{mauve}{rgb}{0.58,0,0.82}
%! endregion

%! region Styling
% Set highlighting color.
\sethlcolor{yellow}

% Set spacing between table and caption.
\captionsetup[table]{skip=8pt}

% Style subsections.
\titleformat{\subsection}{\normalfont\fontsize{12}{15}}{\thesubsection}{1em}{}

% Python code styling.
\lstset{frame=tb,
    language=Python,
    aboveskip=3mm,
    belowskip=3mm,
    showstringspaces=false,
    columns=flexible,
    basicstyle={\small\ttfamily},
    numbers=none,
    numberstyle=\tiny\color{gray},
    keywordstyle=\color{blue},
    commentstyle=\color{gray},
    stringstyle=\color{mauve},
    breaklines=true,
    breakatwhitespace=true,
    tabsize=3
}

% Header and Footer Styles.
\fancypagestyle{firstpage}
{
    \fancyhead[L]{}
    \fancyhead[R]{ \includegraphics[width=2cm,height=2cm,keepaspectratio]{logo}}
    \fancyfoot[R]{Page \thepage \hspace{1pt} of~\pageref{LastPage}}
}

\fancypagestyle{pages}
{
    \fancyhead{}
    \fancyfoot[R]{Page \thepage \hspace{1pt} of~\pageref{LastPage}}
    \addtolength{\topmargin}{-16.34447pt}
}
%! endregion

%! region Commands/Variables
\renewcommand{\UrlFont}{\bfseries}
\newcommand{\var}[1]{\texorpdfstring{#1}\xspace}
\newcommand{\answer}[1]{\textbf{Answer: }\par#1}
\newcommand{\versus}{vs.}
\newcommand{\xtrain}{\var{$X_{train}$}}
\newcommand{\xtest}{\var{$X_{test}$}}
\newcommand{\question}[1]{\section{Question \thesection: \var{\normalfont\normalsize{{#1}}}} \label{sec:question\thesection}}
\newcommand{\subquestion}[1]{\subsection{#1} \label{subsec:question\thesubsection}}
%! endregion

\author{Alan Szmyt}

\title{MET CS677 Data Science with Python \\ Music Genre Classification }

%! endregion
\begin{document}

    % Show all references from bib file without needing to cite them.
    \nocite{*}

    % Create title from the title macro.
    \maketitle

    % Use the fancy style for first page only and switch styles for the rest.
    \thispagestyle{firstpage}

    % Overview of the assignment.
    \subfile{resources/report/overview}

    % Style to set for the rest of the pages.
    \pagestyle{pages}

    I created a python class \textit{ModelAnalyzer} that uses the strategy design pattern to
    switch between machine learning models and datasets to perform training, testing,
    and gather analytics for different types of models.
    These analytics can then be compared to determine a suitable model for predicted music genre.

    I chose to compare a Linear Support Vector Machine (SVM), a Random Forest
    classifier, logistic regression model, and k-Nearest Neighbors (kNN) classifier.
    I also used a Linear Regression model to plot the highest correlation features
    for each genre.

    \jupynotex[8]{project.ipynb}

    Before diving into the model analysis, I first plotted the correlation matrix for
    the entire track dataset to get a high level overview of the correlations between
    all the track features.

    I also created a radar (or spider) plot of the track features for each genre to see
    the distribution of values per genre.
    The radar plot shows that the \textbf{Metal} genre has a high energy and loudness
    and every other value is relatively low.
    For the \textbf{Country} genre, there is a higher danceability value and valence,
    but all the values are below 0.6, so there isn't as much of a weight on loudness
    and energy compared to the \textbf{Metal} tracks.

    \jupynotex[9]{project.ipynb}

    \jupynotex[10]{project.ipynb}

    I started off by training and testing a Linear SVM model and computing the accuracy
    and confusion matrix.
    The Linear SVM performed pretty well with a 93.78\% accuracy.

    \jupynotex[11]{project.ipynb}

    \jupynotex[12]{project.ipynb}

    I then ran training and testing on multiple Random Forest classifiers to determine
    the best resulting hyperparameter values for \textit{trees} and \textit{max\_depth.}

    \jupynotex[13]{project.ipynb}

    \jupynotex[14]{project.ipynb}

    The random forest classifier with the best performance had high accuracy of 97.33\%.

    \jupynotex[15]{project.ipynb}

    \jupynotex[16]{project.ipynb}

    \jupynotex[17]{project.ipynb}

    \jupynotex[18]{project.ipynb}

    \jupynotex[19]{project.ipynb}

    Comparing the four machine learning models resulted with the random forest
    classifier being the best performer for both \textit{accuracy} and \textit{f1 score}.

    \jupynotex[20]{project.ipynb}

    \jupynotex[21]{project.ipynb}

    \jupynotex[22]{project.ipynb}

    For the \textbf{Metal} tracks, the highest correlating features are \textit{energy}
    and \textit{loudness} and the plot of the linear regression fitted model shows that
    the higher energy values correspond to higher loudness (in decibels) values.

    \jupynotex[23]{project.ipynb}

    \jupynotex[24]{project.ipynb}

    \jupynotex[25]{project.ipynb}

    \jupynotex[26]{project.ipynb}

    \jupynotex[27]{project.ipynb}

    Overall, it seems that the \textbf{Metal} tracks have higher \textit{loudness} and
    \textit{energy} and lower \textit{danceability} values, so it seems that are the
    features that distinguish the values compared to the \textbf{Country} tracks.

    \jupynotex[28]{project.ipynb}

    %! region Bibliography
    \bibliography{resources/report/references}
    \bibliographystyle{plain}
    %! endregion

\end{document}
